\documentclass[12pt,a4paper,Flow]{report}
\usepackage[left=2.8cm,right=2.8cm,top=2.5cm,bottom=2cm]{geometry}
\usepackage[BoldFont]{xeCJK}
\usepackage{xeCJK}
\usepackage{graphicx}
\usepackage{float}
\usepackage{indentfirst}
\usepackage{fancyhdr}
\usepackage[toc,header,title]{appendix}
\usepackage{color,xcolor}
\usepackage{listings}
\usepackage{tikz}
\defaultfontfeatures{Mapping=tex-text}
\definecolor{llgray}{rgb}{0.9,0.9,0.9}
\lstloadlanguages{[ansi]c}
\lstset{
  language=python,tabsize=4,keepspaces=true,
  xleftmargin=2em,xrightmargin=2em,aboveskip=1em,
  frame=none,
  commentstyle=\color{red}\itshape, % blue comments
  keywordstyle=\color{blue}\bfseries,
  backgroundcolor=\color{llgray},
  breakindent=22pt,
  numbers=left,stepnumber=1,numberstyle=\tiny,
  basicstyle=\footnotesize,
  showspaces=false,
  showstringspaces=false, % show explicit string spaces
  flexiblecolumns=false,%true,
  breaklines=true,breakautoindent=true,breakindent=4em,
  escapeinside={/*@}{@*/}
}
\usepackage[a4paper,CJKbookmarks,bookmarks=true,bookmarksopen=true]{hyperref}%很强大的一个东西
\hypersetup{
  pdftitle={},
  pdfauthor={Wang Pei},
  pdfkeywords={},
  bookmarksnumbered,
  breaklinks=true,
  pdfview=FitV,       % Or try pdfstartview={FitV}, This lead to uncorre
  urlcolor=cyan,
  colorlinks=true,
  citecolor=magenta,          %citeref's color
  linkcolor=blue,
}
\usepackage{titlesec}
\titleformat{\chapter}{\centering\huge}{\textbf{第}\thechapter{} \textbf{章}}{1em}{\textbf}
\renewcommand\contentsname{目录}
\renewcommand{\appendixpagename}{附录}
\renewcommand{\appendixname}{附录}
\setmainfont{DejaVu Sans}
\setCJKmainfont[BoldFont=WenQuanYi Micro Hei]{SimSun}

\begin{document}
\title{\textbf{体系结构实习——UniCore2模拟器\\实习报告}}
\author{张番栋 00848180\\刘澜涛 00848200\\王 沛 00848205\\CS08}
\date{\today}
\maketitle
\tableofcontents
\newpage
\chapter{实习内容}
\section{目标}
实习的主要内容是用C语言编写一个支持UniCore2精简指令系统的模拟器,并对其进行测试、验证。在之后的报告中,均称实习所完成的模拟器为MinSim。MiniSim的首要目标是进行正确的功能模拟,在此基础上增加对CPU流水线以及Cache的结构模拟,另外还有对程序动态运行情况的统计。MiniSim采用五级流水结构,和真实的UniCore2处理器并不相同。模拟的Cache则具有较灵活的定制性。\\
\section{具体实现功能}
\indent MiniSim支持的UniCore2指令子集中包含五类指令:
\begin{itemize}
\item 数据处理指令
\item 乘法和乘加指令
\item 跳转切换指令
\item 单数据传输指令
\item 条件转移指令和带链接条件转移指令。
\end{itemize}
\indent 从中可以看出,MiniSim并未对处理器特权状态的相关功能进行模拟,进而无法完全支持基于现代操作系统下的大部分程序。针对这个问题,我们做了一些工作,使得MiniSim在接受标准ELF文件作为输入的情况下,仍然能够完成大部分应用级别的功能模拟。\\
\indent 本次实习是与编译实习联合进行,因此我们在完成实习的过程中做了一些编译器和模拟器之间的协调工作。\\
\indent 最后,为了方便使用和调试,我们为MiniSim编写了一个简单的控制台模块,使得MiniSim在运行时可以设置断点,单步执行、查看寄存器、内存和流水线的状态。
\chapter{模拟器的存储体系}
\section{寄存器堆}
MiniSim以UniCore2处理器为目标进行模拟,因此寄存器堆的结构基本与之相同。有所区别的是,由于MiniSim无法处理异常(Exception),也不区分处理器状态,所以不需要SMSR标志寄存器。因此MiniSim共有32个通用寄存器和一个CMSR标志寄存器。
\section{内存管理}
MiniSim在运行时需要在自己的地址空间内同时维护目标程序的地址空间,因此需要建立一套面向目标程序的虚拟内存机制。在这个问题上,MiniSim采用的是页式内存管理机制。对于目标程序的32位地址空间,MiniSim进行了如下划分
\begin{itemize}
\item 31-25 位: 一级页表偏移
\item 24-15 位: 二级页表偏移
\item 14-0 位: 页内偏移
\end{itemize}
\begin{figure}[H]
  \centering
  \includegraphics[width=0.5\textwidth]{pt.png}
\end{figure}
这样的结构表明页的大小为32KB。一级页表常驻内存,共有128个页表项,每个页表项大小为4B,即指向二级页表的指针。每个二级页表含有1024个页表项,每个页表项8B,存储的信息包括其所指向的页的起始地址和该页的读、写、执行属性。\\
\indent 这是一个很经典的虚拟内存管理机制,尤其在管理目标程序的栈时可以提供较好的局部性。在这种机制下,只要模拟器的存储管理模块要提供读、写内存的操作接口,真正执行目标程序的模块就可以减少很多工作量。同时,页式结构在一定程度上阻止非法的内存访问。
\section{Cahce}
MiniSim的Cache采取可定制结构,即用户可通过提供不同参数来获得不同的Cahce组织方式和Cache维护策略。我们共提供3种Cache更新策略(LRU,随机,轮转)和2种Cache写回策略(直写和回写)。考虑到软件实现的Cache对程序性能提高只会起到反作用,因此在我们的数据结构中Cache并不真正保存数据,有的只是Cache本身的状态信息。不管命中与否,数据都要从虚拟内存中取回。\\
\indent 由于Cache命中与否对程序性能影响较大,我们假定在访存时如果Cache命中则一个时钟周期可取回数据;如果失效则需要十个时钟周期。
\chapter{模拟器运行环境的建立}
MiniSim接受标准ELF文件作为输入。在真正模拟运行目标机程序之前,需要做一些初始化工作。换言之,MiniSim需要完成一部分装载器和操作系统的工作。主要内容有ELF文件解析
内存环境的建立、代码段和数据段的载入、寄存器堆和流水线的初始化等等。
\section{对系统相关指令、功能的处理}
由于MiniSim不支持操作系统级别的指令,同时也不需要支持复杂的程序运行环境,所以模拟区间仅限于main函数。也就是说,MiniSim会跳过main函数之前的代码段,待main函数结束后退出。另外,为了便于检查,MiniSim还需要支持整型数据的非格式化输出。输出是与系统调用有关的功能,因此需要做一些约定:在给予MiniSim的输入中,输出功能要被封装在一个特定的函数中。当模拟进行到这个函数时,模拟器会以宿主机的输出调用替代之,然后结束该函数,从返回地址继续执行。为了达到这个目的,需要在ELF文件装载阶段进行一些工作。
\section{ELF文件解析}
需要做的工作主要有两点,具体的流程与ELF文件结构关系密切,这里只进行简单讨论。(ELF文件结构在elf.h头文件中定义。)
\subsection{关键函数的入口地址}
关键函数指的是main函数和输出封装函数。前者的入口是模拟程序的入口,后者的入口则决定了模拟器调用宿主输出机制的时机。大致流程如下:
\begin{enumerate}
\item 找到.shstrtab节区。
\item 遍历所有节区, 找到类型为 SHT\_SYMTAB 的两个节区:.symtab和.strtab。
\item 遍历.symtab节区, 找到需要的函数符号。
\item 返回其虚拟地址。
\end{enumerate}
\subsection{装载程序段}
获得ELF文件中代码段和数据段的位置和大小,以备建立模拟内存环境之用。
\begin{enumerate}
\item 获取程序段表 (Program Header Table) 基址
\item 遍历程序段表, 找到所有类型为 PT\_LOAD 的段。一般只有2个,分别存放代码、只读数据和可读写数据。
\item 根据段的大小和段需要的起始虚拟地址开辟新页,将段的数据从文件装入内存。
\end{enumerate}
基于效率的考虑,MiniSim没有建立动态装载的机制,即在模拟开始前将可能需要的代码和数据一次性装入。如果要进行动态装载,那么在每次访存过程中都需要判断是否要转入新页。由于取值操作每个周期都要进行,这样的判断会带来较大的性能损失。
\subsection{寄存器堆的初始化}
寄存器堆中有一些重要的寄存器需要在执行程序前进行必要的初始化。需要初始化的寄存器主要有4个:
\begin{enumerate}
\item PC。在运行程序前,需要在PC中装入之前解析ELF文件所得到的main函数入口地址。
\item LR。为了让MiniSim能够检测到main函数的结束,需要给LR一个特殊的值(一般是指向保留的地址空间,MiniSim在实现时采用了0这个特殊值)。这样当PC等于这个特殊值是,MiniSim就可以确定main函数已经结束,从而结束对目标程序的模拟执行。
\item SP。该寄存器的初始值决定了目标程序的栈的起始位置。在MiniSim的视线中,SP被初始化为0xf0000000。
\item FP。该寄存器的初始值决定了main函数栈帧的基址,因此需要给予与SP相同的初始值。
\end{enumerate}
\chapter{流水线设计}
\section{整体结构}
\subsection{设计描述}
MiniSim在完成指令功能模拟的同时,以软件的形式模拟了处理器的流水线结构。MiniSim的流水线分为取指(IF),译码(ID),执行(EX),访存(MEM),写回(WB)五级。由于软件不可能实现真正的并发,所以MiniSim的流水线实际上是以从WB到IF的顺序分步执行的。虽是如此,在真实流水线存在的与并发有关的问题在MiniSim中还是有所体现,比如数据和控制冒险。解决这些问题将是流水线设计和实现的重点。下图为流水线的基本结构:
\usetikzlibrary{positioning}
\usetikzlibrary{shadows}
\usetikzlibrary{shapes}
\begin{center}
  \scalebox{1.1}
           {
             \begin{tikzpicture}[node distance=0.5cm]
               \node[fill=blue!20] (wb){WB};
               \node[fill=white] (wbin)[below =of wb]{数据传递};
               \node[fill=blue!20] (mem)[below =of wbin]{MEM};
               \node[fill=white] (memin)[below =of mem]{数据传递};
               \node[fill=blue!20] (ex)[below =of memin]{EX};
               \node[fill=white] (exin)[below =of ex]{数据传递};
               \node[fill=blue!20] (id)[below =of exin]{ID};
               \node[fill=blue!20] (ctrl)[right= of id]{控制};
               \node[fill=white] (idin)[below = of id]{数据传递};
               \node[fill=blue!20] (if)[below =of idin]{IF};
               \node[fill=red!40] (reg)[left= of wb]{寄存器堆};
               \node[fill=red!40,node distance=2cm] (memory)[right = of mem]{内存};
               \node[fill=red!40] (fwd) [left=of id]{前递单元};
               \draw[<->] (memory) -- (mem);
               \draw[<-] (fwd) -- (reg);
               \draw[->] (fwd) -- (id);
               \draw[->] (id) -- (ctrl);
               \draw[->] (ctrl) |- (exin);
               \draw[->] (wbin) -- (wb);
               \draw[->] (mem) -- (wbin);
               \draw[->] (memin) -- (mem);
               \draw[->] (ex) -- (memin);
               \draw[->] (exin) -- (ex);
               \draw[->] (id) -- (exin);
               \draw[->] (idin) -- (id);
               \draw[->] (if) -- (idin);
               \draw[<-] (if) -| (fwd);
               \draw[->] (ex) -- ++(-4,0) -- ++(0,-1) |-  (fwd);
               \draw[->] (mem) -- ++(-3.5,0) -- ++(0,-1) |- (fwd);
               \draw[->] (wb) -| (reg);
             \end{tikzpicture}
           }
\end{center}
\subsection{程序实现}
下面是MiniSim实现流水线的主要数据结构:
\begin{lstlisting}[language=c]
  typedef struct
  {
    FwdData ex_fwd[2];
    FwdData mem_fwd;

    /* data transfered between pipeline stages */
    ID_input id_in;
    EX_input ex_in;
    MEM_input mem_in;
    WB_input wb_in;
  } PipeState;

  typedef struct
  {
    uint32_t reg[32];
    PSW CMSR;
    uint32_t usr_trap_no;
  } RegFile;
\end{lstlisting}
PipeState中存储了每一个流水级所需要的数据及控制信息。
\begin{enumerate}
\item ID\_input,主要是需要译码的指令以及从IF阶段上传的流水级信息。
\item EX\_input,包括ID阶段获取的操作数、需要执行的ALU或乘法器操作类型、以及在ID阶段产生、需要上传个MEM和WB的控制信号。
\item MEM\_input,包括从前一流水级上传的控制信号,主要有读写控制、地址选择(基址还是变址)、访存数据宽度、是否进行对load数据进行符号扩展等。
\item WB\_input,包括需要回写的目标寄存器号,需要回写的数据本身,以及回写目标选择(回写存储器输出,回写EX阶段输出,或者都回写,或者都不回写)。
\end{enumerate}
\section{数据冒险相关}
由于数据冒险和控制冒险问题影响着整个流水线的运行,所以在描述各个流水级之前,先对MiiSim解决数据冒险的机制和策略做一介绍。
\subsection{两种数据冒险}
在MiniSim中,影响流水线运行的数据冒险实际上只有RAW一种。根据解决方式的不同,将之划分为两种:
\begin{enumerate}
\item 第一种RAW数据冒险是由数据处理指令产生的。在这种情况下,可以对数据处理指定的结果进行前递来满足后续指令的数据需求。
\item 第二种RAW数据貌相是由加载指令产生的。因为加载指令的结果在MEM阶段才会产生,所以无法通过前递满足紧随其后的一条指令的需求。如果此时发生数据相关,只能通过暂停流水线来解决。我们称这种情况为加载互锁。
\end{enumerate}
\subsection{数据冒险的解决}
\begin{enumerate}
\item 数据前递。数据前递的实现方法是在流水线状态信息(数据结构为PipeState)的ID\_input域中加入了三个前递数据槽。其中两个接受EX流水级前递的数据,一个接受MEM流水级前递的数据。为EX设立两个数据槽的原因是,UniCore2指令集中的访存指令对基址寄存器可以进行回写。因此相邻的两条指令中可能产生多达4次的回写。同时Unicore32指令最多可以有3个源寄存器。基于以上事实可知,仅仅两个数据槽不能满足所有的前递要求。因此需要为EX阶段增加一个数据槽。MEM阶段因为距离WB阶段只有1个周期,因此只需要有一个数据槽即可。
\item 加载互锁。加载互锁的实现比较简单,只需在ID阶段判断是否前一个流水级是否为load指令,该load指令与当前ID阶段的指令是否有数据相关即可。发生相关的条件是要读取的寄存器号与需要回写的寄存器号相同。由于流水线和模拟器流程执行顺序的关系,ID阶段要判断的回写信息存储在流水线状态中的mem\_in一级。当然,如果ID之后的流水级中插入了气泡,那么判断无需暂停流水线。
\end{enumerate}
\indent 因此对于数据冒险问题的解决可以概括如下:\\
\indent 在数据前递方面,是否进行前递基于对回写信号的解析。为EX设立的两个数据槽形成一个队列,第一个数据槽永远保存最新一次的前递信息。\\
\indent 读寄存器堆方面,首先判断是否需要进行加载互锁,如果是,暂停流水线,否则要扫描前递槽。需要注意的是EX数据前递槽的第二个槽要最后扫描,如果命中则取数据。未命中,则直接从寄存器堆取数据。
\section{取值(IF)阶段}
IF阶段完成的工作是取值,判断关键入口并进行处理,或者判断模拟是否已经结束。
\subsection{取指令}
IF阶段中会根据寄存器堆中PC的值访问存储器(借口由存储管理模块提供),并在取值结束后对PC进行更新(加4操作)。\\
\indent 值得注意的是,由于UniCore2指令系统中的PC对汇编程序是可见的,所以程序员可以将PC作为数据处理指令的目的寄存器使用。由于每一次PC的非累加性更改都相当与一次跳转,所以我们希望对与PC的修改可以尽早被探测到。因此,需要考虑PC的前递问题。\\
\indent 这也就是说,在取指令时不能依赖当前的PC值,而是需要先检查前递单元中是否有指向PC的前递。如果有,则要根据前递值去取指,并根据前递来源来清空流水线。如果前递来自EX阶段,则需要清空从ID到MEM的所有流水级。需要清空MEM的原因是,由于流水线倒序执行,到IF得到前递值时,制造前递的指令实际上已经从EX流水级进入了MEM。同理,如果前递来自较早的EX阶段或者MEM阶段,需要将从ID到WB的所有流水级排空。\\
\indent 可以看到,这样对PC的操作会对流水线产生非常不利的影响。因此我们约定在编译器中绝对不会通过直接修改PC来实现跳转。
\subsection{对关键函数入口地址的处理}
最初已经提到,我们的模拟器无法处理与处理器特权态有关的指令,加之编译器方面的限制,只能支持一种与系统调用有关的行为,即输出一个整数。在模拟器执行指令前的ELF文件载入工作中已经获得了封装输出子例程的函数入口。IF阶段要做的就是判断当前的PC是否为这个关键入口,如果是,则IF向ID送出一条伪指令,编码为0xffffffff。这条指令在ID阶段会被进行特殊处理。
\subsection{模拟结束的时机}
对结束模拟执行时机的判断也在IF阶段进行。同时将lr寄存器的值设置为0。这样一旦在IF阶段发现PC为全0,就可以认为模拟即将结束了。此时IF阶段会在产生4个汽泡后(带其他流水级中的指令执行完毕),产生结束模拟的信息。
\section{指令译码(ID)阶段}
ID阶段是五个流水级中最复杂的阶段,需要完成的工作较多。对指令类型的判断、重要控制信号以及EX阶段所需操作数的产生、跳转的实现等都在ID阶段完成。
\subsection{指令译码}
对指令进行译码,根据指令信息产生相关的控制信号(在程序中就是为与控制有关的变量赋相应的值)。控制信号由专门的控制模块产生。它独立于ID模块。
\subsection{产生操作数}
EX阶段所需要的两个操作数也在ID阶段产生,所以需要完成读寄存器堆、对寄存器中的值进行移位等工作。这之中伴随着数据冒险的检测和解决。如果检测到需要加载互锁,ID阶段函数会返回一个特殊的值交由流水线判断,流水线控制逻辑会根据ID的返回值决定是否进行互锁。一般真实流水线中取前递值的操作应该是在EX阶段的最初时刻进行。在MiniSim,基于简化软件设计的考虑,我们将数据冒险的检测和解决封装到了读寄存器堆操作中,也就是说读取前递值这项任务在ID阶段就已经完成。\\
\indent EX阶段需要的第二操作数一般并不是某个寄存器中的数据本身,而是需要对其进行某种移位操作。在这过程中还需要根据移位情况对CMSR的C位进行设置。具体的移位类型有:
\begin{enumerate}
\item 寄存器的立即数位移。
\item 寄存器的寄存器位移。
\item 立即数的立即数循环右移。
\end{enumerate}
移位的具体实现被封装在一个函数中,该函数会根据基址,移位值和移位类型自动产生结果并设置标志位。
\subsection{跳转}
MiniSim实现的UniCore2指令子集中共有三种类型的跳转:
\begin{itemize}
\item 跳转切换,以一个寄存器中的值为跳转目标。
\item 无条件跳转,目标地址相对于当前PC值进行计算。
\item 条件跳转,条件(CMSR中各种条件码的组合)满足时跳转到一个PC相对地址。
\end{itemize}
基于流水线效率的考虑,MiniSim中所有跳转都在ID阶段完成,因此跳转的副作用是需要在ID阶段插入一个气泡。这个操作相当于真实处理器中排空IF流水级的操作。具体来讲:\\
\indent 对于跳转切换指令而言,由于其目标地址存放于寄存器中(一般是LR),所以与数据相关的一系列问题都需要考虑。也就是说,跳转切换的正确执行依赖于数据前递和加载互锁机制。\\
\indent 无条件跳转的目标地址由当前PC与一个偏移量相加得来。因此,该类跳转同样要关注前递的数据信息。比较特别的是,由于PC不能作为装载指令的目的寄存器,因此无条件跳转无需关注加载互锁的问题。\\
\indent 条件跳转与无条件跳转相比较,多出了条件判断的步骤。当前条件的来源最近有可能是前一条指令。由于MiniSim的流水级以倒序执行,因此在ID阶段的指令无需关注EX阶段是否已经写回条件码。另外MiniSim没有另外提供条件跳转预测的机制,因此也可以认为MiniSim总是预测跳转不发生。而一旦跳转发生,产生气泡的工作就必不可少。
\section{指令执行(EX)阶段}
该流水级进行的操作是执行ALU和乘法器运算,根据需要还可能会对CMSR的标志位进行修改。由于操作数和操作符都由ID阶段提供,因此该阶段除了要进行计算并将计算结果传输到下一个流水级之外,只需要考虑数据的前递问题。对于EX阶段的前递任务而言,需要额外关注的是它拥有两个前递数据槽。这一点已经在数据冒险部分说明过。维护这两个槽组成的队列是EX阶段的主要任务之一。另外,由于乘法操作耗时较长,在进行数据统计时要增加额外的计数(简单起见就不进行真正的流水线暂停了)。
\section{访存(MEM)阶段}
与EX阶段相同,决定MEM阶段行为的所有控制已经在ID阶段生成,MEM阶段本身只需要按照这些控制的指示进行访存,并将访存结果传输到下一流水级即可。访存需要的行为都由存储管理模块提供接口,MEM阶段唯一额外的工作是需要对数据前递的相关信息进行一些必要的维护。
\section{写回(WB)阶段}
这是模拟器行为最为简单的一个流水级,所需要的工作只是根据前一流水级传递而来的信息进行寄存器堆的写回操作而已。写回的目标寄存器选择有四种可能:
\begin{enumerate}
\item 无写回。
\item 写回ALU或者乘法器输出。
\item 写回从存储器读取的数据。
\item 上面两种数据均需写回。发生这种情况是由于访存基址写回这一特性。
\end{enumerate}
\chapter{模拟器控制台}
\section{简介}
出于方便调试和使用的考虑,MiniSim实现了一个类似于GDB的简单控制台。实现的功能主要有:
\begin{enumerate}
\item 设置断点。考虑到断点的数目不会很多,MiniSim利用二分搜索树存储和查找断点。据实际测试,性能上可以满足要求。
\item 单条指令执行。在进行数据相关调试时非常有用的功能。
\item 输出当前模拟器信息。输出的信息主要有
\begin{itemize}
\item 寄存器堆状态,包括32个用户可见寄存器的值以及CMSR的各个标志位状态。
\item 当前每个流水级将要运行的指令PC以及指令编码。
\item 栈中存储的信息。
\end{itemize}
\item 直接退出。允许用户在模拟过程中直接结束模拟。
\item 模拟重启。允许用户重新从目标程度的main函数入口开始模拟运行目标程序。
\end{enumerate}
由于缺少反汇编功能,控制台对用户并不十分友好,但对于想细致了解MiniSim模拟行为的用户还是有一定帮助的。
\section{操作说明}
下面给出控制台的控制命令:
\begin{itemize}
\item b <hex>,在地址为<hex>的指令处设置一个断点。<hex>为一个在32位地址空间内的16进制数。
\item c,模拟运行继续,直至遇到下一个断点后暂停。
\item d <hex>,删除在<hex>处设置的断点(如果有)。
\item n,模拟器运行一个周期后暂停。
\item pr,打印当前寄存器堆信息。
\item pp,打印当前各个流水级信息。
\item ps <hex1> [<hex2>],以字为单位打印栈的内容,范围是从<hex1>到<hex2>,如果<hex2>未提供,则只打印地址为<hex1>的数据,大小仍为一个字。
\item q,结束模拟,程序退出。
\item r,重新运行目标程序。
\end{itemize}
\chapter{项目测试与验证大纲}
对于MiniSim的测试和验证主要有三部分:指令级的验证,与流水线有关的验证(数据、控制相关),对整个模拟流程的验证以及通过模拟运行完整程序进行整体测试。
\section{指令级测试}
这部分的测试主要是为了验证模拟器各个模块的功能是否正确和完备,验证的方法是编写汇编测试程序,每个测试程序中有一条关键指令,这条指令是测试需要关注的重点。测试通过的标准是:指令总体行为正确并且各个流水级在处理指令时的行为正确。在选择测试样例时遵循下面的原则:
\begin{enumerate}
\item 要覆盖大的指令类别。指令公分五大类,报告之初已列举。
\item 要覆盖每个特殊功能点。比如跳转指令的链接行为,访存指令的post/pre-indexed寻址,基址写回行为,跳转的不同条件等。
\item 两者的适当组合。
\end{enumerate}
\section{数据、控制相关测试}
测试程序中包含了存在数据相关的指令序列,用于验证数据前递和加载互锁机制的有效性。另外通过循环、分支等结构验证流水线是否存在控制冒险。
\section{对模拟流程的测试}
主要内容是测试模拟器能否正确找到目标程序的main函数入口,能否正确退出,能否正确截取特殊函数调用(主要与输出相关)。
\section{运行完整程序}
运行以C语言编写、经Unicore32工具链生成的ELF文件。主要用于测试复杂语言特性能否被正确实现,如数组、函数递归等。典型的测试程序有冒泡排序、快速排序、大整数加法等。
\chapter{使用说明}

\end{document}
