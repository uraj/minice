\documentclass[12pt,a4paper,Flow]{report}
\usepackage[left=2.8cm,right=2.8cm,top=2.5cm,bottom=2cm]{geometry}
\usepackage[BoldFont]{xeCJK}
\usepackage{xeCJK}
\usepackage{graphicx}
\usepackage{float}
\usepackage{indentfirst}
\usepackage{fancyhdr}
\usepackage[toc,header,title]{appendix}
\usepackage{color,xcolor}
\usepackage{listings}
\usepackage{tikz}
\defaultfontfeatures{Mapping=tex-text}
\definecolor{llgray}{rgb}{0.9,0.9,0.9}
\lstloadlanguages{[ansi]c}
\lstset{
  language=python,tabsize=4,keepspaces=true,
  xleftmargin=2em,xrightmargin=2em,aboveskip=1em,
  frame=none,
  commentstyle=\color{red}\itshape, % blue comments
  keywordstyle=\color{blue}\bfseries,
  backgroundcolor=\color{llgray},
  breakindent=22pt,
  numbers=left,stepnumber=1,numberstyle=\tiny,
  basicstyle=\footnotesize,
  showspaces=false,
  showstringspaces=false, % show explicit string spaces
  flexiblecolumns=false,%true,
  breaklines=true,breakautoindent=true,breakindent=4em,
  escapeinside={/*@}{@*/}
}
\usepackage[a4paper,CJKbookmarks,bookmarks=true,bookmarksopen=true]{hyperref}%很强大的一个东西
\hypersetup{
  pdftitle={},
  pdfauthor={Wang Pei},
  pdfkeywords={},
  bookmarksnumbered,
  breaklinks=true,
  pdfview=FitV,       % Or try pdfstartview={FitV}, This lead to uncorre
  urlcolor=cyan,
  colorlinks=true,
  citecolor=magenta,          %citeref's color
  linkcolor=blue,
}
\usepackage{titlesec}
\titleformat{\chapter}{\centering\huge}{\textbf{第}\thechapter{} \textbf{章}}{1em}{\textbf}
\renewcommand\contentsname{目录}
\renewcommand{\appendixpagename}{附录}
\renewcommand{\appendixname}{附录}
\setmainfont{DejaVu Sans}
\setCJKmainfont[BoldFont=WenQuanYi Micro Hei]{SimSun}

\begin{document}
\title{\textbf{编译实习——minic编译器\\项目文档}}
\author{张番栋 00848180\\刘澜涛 00848200\\王 沛 00848205}
\date{}
\maketitle
\tableofcontents
\newpage
\chapter{实习内容}
实习的内容是以C语言编写一个以C语言子集(称为MiniC)为源程序,目标机为UniCore2体系结构的编译器。报告后面的篇幅中,均称实习中编写的编译器为minicc。MiniC的语法与C语言基本相同,比较重要的区别是不支持除法、浮点计算、多重指针和高维数组。\\
\indent 实习对minicc的基本要求是,可以对给定的源程序进行语法和语义上的正确性检查并能够输出正确的汇编程序代码,最终与体系结构实习所完成的模拟器协同工作。这一过程需要的预处理器、库以及二进制工具由外部提供。
\chapter{编译器结构与实现}
与传统编译器一样,minicc被划分为前端和后端两部分。由于minicc的优化策略不多,且分布比较零散,所以会将优化单独作为一部分进行描述。
\section{前端}
minicc前端的任务有
\begin{enumerate}
\item 针对源程序进行词法、语法分析。
\item 在语法分析的过程中构建抽象语法树、建立符号表。
\item 针对语法树进行语义检查,包括类型检查与左值存在性检查。
\item 根据语法树生成中间代码。
\end{enumerate}
下面对各项工作进行详细说明。
\subsection{词法、语法分析}
minic的词法和语法分析程序由flex和bison辅助生成。\\
\indent 就词法分析而言,由于MiniC的文法并没有对源程序中字符、字符串的形式做规定,因此在实现时,我们只支持了有限的一部分。对于形如“\\x[0-9a-z][0-9a-z]”的字符,词法分析并不识别。\\
\indent 语法分析方面,我们对原始的MiniC文法进行了一定修改,使其支持悬空if控制流。另外通过规定运算符优先级的方式避免的文法规约时的二义性。
\chapter{测试与验证}
\chapter{编译功效}
\end{document}
